\documentclass[12pt, titlepage]{article}
%Packages
\usepackage{
  acronym,
  enumerate,
  fancyhdr,
  hyperref,
  indentfirst,
  lastpage,
  listings,
  microtype,
}
\usepackage[top=1in, bottom=1in, left=1in, right=1in]{geometry}
\usepackage[htt]{hyphenat}

% declare some useful macros
\newcommand{\assignmentNumber}{Assignment 4}
\newcommand{\courseName}{CS 444}
\newcommand{\z}[1]{\texttt{#1}}

\acrodef{jls}[JLS]{Java Language Specification}

\lstset{basicstyle=\ttfamily}

\begin{document}
\pagestyle{fancyplain}
\thispagestyle{plain}

%Headers
\fancyhead{}
\fancyfoot{}
\rhead{\fancyplain{}{Page \thepage\ of \pageref*{LastPage}}}
\chead{\fancyplain{}{\assignmentNumber}}
\lhead{\fancyplain{}{\courseName}}

\title{CS444 - \assignmentNumber\\Name Resolution, Type Checking, and Static Analysis}
\date{\today}
\author{Alex Klen\\20372654\and Sanjay Menakuru\\20374915\and Jonathan Wei\\20376489}

\maketitle

\section{Introduction}

We decided to tackle assignments 2, 3, and 4 together, as we realized that all
of the necessary computation for these three assignments could be carried out
in the same set of AST traversals. Our resulting code is structured in five
phases, described at a high level in the \nameref{sec:arch} section, with more
detail in the \hyperref[sec:assg]{subsequent} section.

\section{Architecture}\label{sec:arch}

\subsection{Phases of the compiler}

We will consider only phases of the compiler that were added for assignment 2
through assignment 4. These phases are the following:
\begin{enumerate}
  \item \nameref{subsubsec:col-types}
  \item \nameref{subsubsec:fields}
  \item \nameref{subsubsec:type-checking}
  \item \nameref{subsubsec:const-prop}
  \item \nameref{subsubsec:data-flow}
\end{enumerate}

Here is a mapping between the various sections of the assignments and the 5
phases above:
\begin{enumerate}[\bf{A}1\ \ ]
  \setcounter{enumi}{1}

  \item
  \begin{description}
    \item[Environment Building]
      \hyperref[subsubsec:col-types]{Phase 1}, \hyperref[subsubsec:fields]{phase 2},
      and \hyperref[subsubsec:type-checking]{phase 3}
    \item[Type Linking]
      \hyperref[subsubsec:col-types]{Phase 1} and \hyperref[subsubsec:fields]{phase 2}
    \item[Hierarchy Checking]
      \hyperref[subsubsec:fields]{Phase 2}
  \end{description}

  \item
  \begin{description}
    \item[Name Disambiguation]
      \hyperref[subsubsec:type-checking]{Phase 3} and \hyperref[subsubsec:data-flow]{phase 5}
    \item[Type Checking]
      \hyperref[subsubsec:type-checking]{Phase 3}
  \end{description}

  \item
  \begin{description}
    \item[Reachability Checking]
      \hyperref[subsubsec:const-prop]{Phase 4} and \hyperref[subsubsec:data-flow]{phase 5}
  \end{description}
\end{enumerate}

\subsubsection{Collecting Types}\label{subsubsec:col-types}

This phase walks over all compilation units and collects all type names. This
is done in \z{types/types.cpp}. It adds all packages and types in a particular
compilation unit and adds them to a \z{TypeSet}.

A \z{TypeSet} is a data structure that maps a qualified name to a type. The
public API is in \z{types/typeset.\{h,cpp\}}, and the implementation is in
\z{types/typeset\_impl.\{h,cpp\}}. A \z{TypeSet} supports scoped-lookup; for instance,
it has a method called \z{WithImports} that provides a `view' into the \z{TypeSet}
assuming the provided imports are in scope. Other such methods include
\z{WithPackage} and \z{WithType}. As future phases recurse through the AST, they
call the appropriate methods on \z{TypeSet} to obtain the correct scoped view of
visible types.

When building a \z{TypeSet}, we verify that each qualified name refers to a
unique class, or a package, but not both. We emit an error for each violation
of this constraint, and suppress any future errors referencing this type. This
is done in the \z{TypeSetBuilder::Build} method which is located in
\z{types/typeset.cpp}.

\subsubsection{Collecting Fields and Methods}\label{subsubsec:fields}

First, this phase walks through the body of every type and records information
about all fields and methods. This information is stored in a class named
\z{TypeInfoMap}, which can be found in \z{types/type\_info\_map.\{h,cpp\}}.

For fields, we record its name, modifiers, containing class, and type.

For methods, we record its name, modifiers, parameter types, containing class,
return type, and a flag \z{is\_constructor}.\\

Next, this phase verifies that the implements-extends graph is acyclic and
well-formed. By well-formed, we mean that it verifies that no interface extends
a class, no class implements another class, and no class extends an interface.
The cycle-checker also produces a topological ordering of the types that
guarantees that for all types $T$, $T$'s parents are visited before $T$
itself.\\

Finally, this phase `pushes down' all method and fields from parents to
children. We associate with each type a \z{FieldTable} and a \z{MethodTable}. These
contain the types' members, and all inherited members.

When pushing down members, we validate several rules of Joos; here is a
non-exhaustive list of these rules:
\begin{enumerate}
  \item An inherited method cannot lower the visibility of a method in a parent
  class.

  \item A parent class must contain a zero-argument constructor.

  \item A class with abstract methods must be declared abstract.

  \item A class may not have multiple fields with the same name.

  \item A class may not have multiple methods with the same signature (where
  the signature includes the name and the argument types).
\end{enumerate}

\subsubsection{Type Checking}\label{subsubsec:type-checking}

This phase validates that the program obeys the Joos typing rules. It uses the
\z{TypeSet} and the \z{TypeInfoMap} from the previous two phases. The source code
for this phase can be found in \z{types/typechecker.\{h,cpp\}}. It is defined as a
Visitor implementation that rewrites the AST to have type information
associated with all expressions and declarations.\\

This phase also introduces the \z{SymbolTable} class, which is located in
\z{types/symbol\_table.\{h,cpp\}}. This class manages a simple mapping from a string
to a type. It supports entering and leaving scopes, and enforces Joos's
restriction on variables with overlapping scopes. Specifically, it disallows
variables with identical names overlapping in scope. It also disallows
variables from being referenced in their own initializers. The type checker
implements `alpha-renaming'; that is to say, it gives every variable a unique
identifier to distinguish two variables with the same name in different scopes.\\

Each node in the AST first type checks its children by visiting them. If any of
its children fail to type check or are pruned, then the node will assume that
an error was emitted below, and will prune itself from the AST without emitting
further errors; this pruning will avoid the cascade of errors commonly found in
many compilers.\\

All ambiguous qualified names are handled using the algorithm provided in the
\ac{jls}. Specifically, we first try to resolve the first segment of a
qualified name in the \z{SymbolTable}. Next, we try and parse the first segment
as a field reference in the containing class's \z{FieldTable}. Finally, we try to
match a prefix of the qualified name in the \z{TypeSet}. If we find a match, we
split the qualified name into different AST types that are more specific to the
usage. For instance, we rewrite a local field reference to a \z{FieldDerefExpr}.

The same rules are also applied to all \z{CallExpr}s, with a slight modification.
We simply slice off the last part of the qualified name, and recurse on the
remaining parts. Once the remaining parts have been resolved, we then look for
a method in the \z{MethodTable} of the type returned from the recursion.

For both field references and method calls, we also include an implicit \z{this}
if the qualified name was in an instance context and an explicit \z{this} was
omitted.

We have several important utility methods such as \z{IsCastable},
\z{IsAssignable}, \z{IsReferenceWidening}, and so forth. These methods can be
found in \z{types/typechecker\_utils.cpp}.

\subsubsection{Constant Propagation}\label{subsubsec:const-prop}

In this phase, we rewrite the AST to contain a new type of \z{Expr}, named
\z{ConstExpr}. \z{ConstExpr} is used to mark expressions that can be resolved
at compile time. For instance, the expression `$1 + 1$' would initially be
rewritten with both of the `1's as \z{ConstExpr}s. We then have a further rule
that the addition of two \z{ConstExpr}s is a \z{ConstExpr}. In this way, we
resolve constant subtrees upwards, until we reach an unresolvable node, like a
reference to a local variable.

The code for this can be found in
\z{types/constant\_folding\_visitor.\{h,cpp\}}.  Note that this implementation
is incomplete, and only implements a subset of the behaviour specified in the
\ac{jls}. We plan to complete this implementation before starting work on code
generation, but we found that the subset implemented was sufficient to satisfy
the Assignment 4 Marmoset tests.

\subsubsection{Data-flow Analysis}\label{subsubsec:data-flow}

This phase performs another walk over the AST to check that all non-void
methods return a value, and that all code is reachable. These checks are
implemented as yet another Visitor, which can be found in
\z{types/dataflow\_visitor.\{h,cpp\}}.

The visitor recurses through all methods, maintaining a single flag named
\z{reachable}. This flag is set to \z{false} on a return statement, and an error is
emitted if a statement is visited when this flag is \z{false}. An error is also
emitted if the flag is still \z{true} after visiting the body of a method with
a non-void return type.

The visitor implements special rules for loop statements with constant
conditional expressions. Specifically, it looks for the \z{ConstExpr} nodes
emitted by the previous pass. See the \ac{jls} for these specific reachability
rules.

In addition, this phase verifies that field initializers do not reference
fields declared later in the source file.

\section{Files}

Here is a mapping from various files in the \z{types} directory to the various
phases above.

\begin{itemize}
  \item \z{types/constant\_folding\_visitor.\{h,cpp\}}: See \hyperref[subsubsec:const-prop]{Phase 4}
  \item \z{types/dataflow\_visitor.\{h,cpp\}}: See \hyperref[subsubsec:data-flow]{Phase 5}
  \item \z{types/decl\_resolver.\{h,cpp\}}: See \hyperref[subsubsec:fields]{Phase 2}
  \item \z{types/symbol\_table.\{h,cpp\}}: See \hyperref[subsubsec:type-checking]{Phase 3}
  \item \z{types/type\_info\_map.\{h,cpp\}}: See \hyperref[subsubsec:fields]{Phase 2}
  \item \z{types/typechecker.\{h,cpp\}}: See \hyperref[subsubsec:type-checking]{Phase 3}
  \item \z{types/typechecker\_errors.cpp}: See \hyperref[subsubsec:type-checking]{Phase 3}
  \item \z{types/typechecker\_utils.cpp}: See \hyperref[subsubsec:type-checking]{Phase 3}
  \item \z{types/typeset.\{h,cpp\}}: See \hyperref[subsubsec:col-types]{Phase 1}
\end{itemize}

\section{Challenges}

\section{Testing}

\end{document}

